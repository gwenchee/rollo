\section{Generative Design}
% What is generative design? How does it work? 
Generative design is an exploratory design method that autonomously 
generates optimal designs by iteratively varying design geometry 
to meet user-defined performance metrics 
\cite{krish_practical_2011,oh_deep_2019}.
The user defines the design parameters and the generative 
design software helps the user create many solutions 
simultaneously \cite{autodesk_autodesk_2020}. 
This sometimes results in unanticipated unique solutions, that 
would have been difficult to discover using traditional methods
\cite{autodesk_autodesk_2020}.
Generative design varies the parameters of the problem definition
\cite{matejka_dream_2018}. 
At each iteration step, the design is evaluated on the 
performance metrics. 
Based on the results, the generative design algorithm changes the 
interval allowed for each design geometry variable, refining 
design constraints (problem definition) and moving towards 
designs that best meets performance metrics.

There is confusion to how generative design differs from other shape 
optimization tools. 
Generative design is more than topology optimization which has been 
around since 1988 \cite{bendsoe_generating_1988}. 
Topology optimization requires the user to start with a complete design 
and the software improves it by removing material, it answers the 
fundamental engineering question of how to place material in a domain space 
to obtain best structural performance \cite{sigmund_topology_2013}.   
Generative design does not require the user to start with a complete design, but 
instead a few design constraints. 
Table \ref{tab:compare} summarizes the differences between generative design and 
traditional design optimization tools. 

\begin{table}[!htbp]
        \caption{Comparison of generative design and traditional design optimization tools
        \cite{autodesk_fusion_2020}.}
        \label{tab:compare}
        \centering
        \doublespacing
        \small
        \begin{tabular}{p{3.7cm}|p{5.5cm}p{6cm}}
        \hline
        \textbf{Point of Comparison} & \textbf{Generative Design} & \textbf{Traditional Design Optimization Tools}  \\ \hline
        Initial input & A few design constraints & Complete design \\ 
        No. of outcomes & Generates a wide set of designs that meets design constraints & Identifies one unique design that meets constraints \\   
        Solving strategies & Uses multiple strategies to solve design problem & Mainly uses topology optimization solutions \\ \hline
        \end{tabular}
\end{table}

% How does it actually work? What models? 
\subsection{Optimization Techniques}
Multi-objective design problems inevitably require a trade off between 
desirable attributes \cite{byrne_evolving_2014,simon_sciences_2019}. 
In nuclear reactor design there are many trade offs, one example is the 
trade-off between neutron economy and fuel enrichment. 
A reactor design must have sufficient neutron economy to ensure criticality, 
but must also have a low fuel enrichment to reduce proliferation risk.
Conflicting objectives means that there is no one perfect solution, but a set
of equally optimal solutions \cite{byrne_evolving_2014}.
Multi-objective problems are difficult to optimize, such problems 
cannot be handled by classical optimization methods such as gradient 
methods, because they tend to only find the local optimum 
\cite{renner_genetic_2003} and are efficient for a narrow subset 
of problems \cite{zames_genetic_1981}. 

Evolutionary algorithms have proven to be successful 
methods to optimize multi-objective problems \cite{krish_practical_2011} as 
they can find a solution near the global optimum \cite{renner_genetic_2003}. 
The most popular evolutionary algorithms used to solve multi-objective 
problems are genetic algorithms 
\cite{byrne_evolving_2014, krish_practical_2011}. 
Genetic algorithms differ from classical optimization techniques in four ways 
\cite{zames_genetic_1981, pereira_genetic_2000}: 
\begin{enumerate}
        \item Work with each parameter's constraints, not
        the parameters themselves. 
        \item Search a population of points, not a single point. 
        \item Does not need prior knowledge about search space and conducts
        optimization process without calculating derivatives, continuity, 
        limits, etc. 
        \item Use stochastic rules, not deterministic rules. 
\end{enumerate}

\subsubsection{Genetic Algorithms}
Genetic algorithms imitate natural selection to evolve solutions 
by (1) maintaining a population of solutions, (2) allowing 
fitter solutions reproduce, and (3) letting lesser fit solutions die off, 
resulting in final solutions that are better than the previous generations 
\cite{renner_genetic_2003}. 
Genetic algorithms demand extra computational cost compared to classical 
optimization techniques. 
Due to the availability of supercomputers, the use of genetic algorithms 
has increased and its practical application has become a reality
\cite{pereira_genetic_2000}. 
Figure \ref{fig:genetic_alg} depicts the iterative process of using a genetic 
algorithm to solve a problem. 
The key part of this process is defining evaluation and termination criteria. 
Evaluation criteria refers to the performance metrics that each solution is 
measured against. 
Termination criteria refers to the range of values for each performance metric
that a solution must meet to be considered `good enough'. 
Attainment of optimum is much less important for complex systems, the goal is 
to get a good solution without sacrificing the level of performance (speed of 
code) \cite{zames_genetic_1981}. 

\begin{figure}[!htbp]
        \centering
        \begin{tikzpicture}[node distance=1.7cm]
                \tikzstyle{every node}=[font=\small]
                \node (1) [lbblock] {\textbf{Create initial population}};
                \node (2) [lbblock, below of=1] {\textbf{Evaluate initial population}};
                \node (3) [loblock, below of=2, yshift = -1.3cm] {\textbf{Create new population:} \\ 
                \begin{enumerate} \item select individuals for mating 
                                  \item create offspring by crossover 
                                  \item mutate selected individuals 
                                  \item keep selected individuals from previous generation
                                 \end{enumerate}};
                \node (4) [loblock, below of=3, yshift=-1.3cm] {\textbf{Evaluate new population}};
                \node (5) [loblock, below of=4] {\textbf{Is termination criteria satisfied?}};
                \node (6) [lbblock, below of=5] {\textbf{Best solution is returned!}};
                \draw [arrow] (1) -- (2);
                \draw [arrow] (2) -- (3);
                \draw [arrow] (3) -- (4);
                \draw [arrow] (4) -- (5);
                \draw [arrow] (5) -- node[anchor=east] {Yes} (6);
                \draw [arrow] (5) -- ([shift={(0.5cm,0cm)}]5.east)-- node[anchor=west] {no} ([shift={(0.5cm,0cm)}]3.east)--(3);
        \end{tikzpicture}
        \caption{Process of solving a problem with genetic algorithm 
        \cite{renner_genetic_2003}. When a population does not meet termination 
        criteria, a new generation is created. This occurs iteratively till the 
        termination criteria is met. }
        \label{fig:genetic_alg}
\end{figure}

Two types of genetic algorithms are typically used to solve shape 
optimization problems: parametric and cell. 
Parametric genetic algorithms optimize complex shapes by varying parametric variables
to meet the desired design performance \cite{von_buelow_paragen_2012}.
Examples of parametric variables are: diameter of a sphere, twist angle of a cylinder, 
etc.  
Optimization of aerodynamic configurations \cite{makinen_multidisciplinary_1999}, 
and truss and bridge structures \cite{raich_evolving_2000} used parametric 
genetic algorithms.
Cell genetic algorithms represent the shape of an object to be optimized in 
small subdivided rectangular domains (pixels in 2D, voxels in 3D) 
\cite{renner_genetic_2003}. 
Figure \ref{fig:cell} shows an example of cell representation. 
Cell genetic algorithms are advantageous compared to parametric genetic 
algorithms since the initial structure and topology of the object need not 
be created and is instead developed through the iterative optimization 
process \cite{renner_genetic_2003}. 
Both parametric and cell genetic algorithms will be explored for the generative 
reactor design problem. 

\begin{figure}[!htbp]
	\begin{center}
		\includegraphics[scale=0.25]{./figures/cell.png}
        \end{center}	
        \caption{Cell representation in 2D \cite{renner_genetic_2003}.}
        \label{fig:cell}
\end{figure}

\subsection{State of Work}
Generative design is used in the following industries: 
automotive \cite{deplazes_autodesk_2019}, aerospace \cite{byrne_evolving_2014},
architecture, civil engineering, construction, product design etc. 
For the automotive and aerospace industries, the goal of generative 
design is to decrease 
the weight of the vehicle while ensuring each part can continue to 
withstand the stresses and strains that are put on it within a 
safety factor. 
These industries can rely on commerical softwares such as 
Autodesk Fusion360 \cite{autodesk_autodesk_2020} and  
SolidWorks Topology \cite{lombard_solidworks_2008}
to produce their generative designs, since these softwares have 
structural simulation-based generative design technology.  

Civil engineering and architecture generative design projects use specific s

